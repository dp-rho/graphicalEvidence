\nonstopmode{}
\documentclass[a4paper]{book}
\usepackage[times,inconsolata,hyper]{Rd}
\usepackage{makeidx}
\usepackage[utf8]{inputenc} % @SET ENCODING@
% \usepackage{graphicx} % @USE GRAPHICX@
\makeindex{}
\begin{document}
\chapter*{}
\begin{center}
{\textbf{\huge Package `graphicalEvidence'}}
\par\bigskip{\large \today}
\end{center}
\ifthenelse{\boolean{Rd@use@hyper}}{\hypersetup{pdftitle = {graphicalEvidence: Graphical Evidence}}}{}
\begin{description}
\raggedright{}
\item[Type]\AsIs{Package}
\item[Title]\AsIs{Graphical Evidence}
\item[Version]\AsIs{1.0}
\item[Date]\AsIs{2024-01-19}
\item[Author]\AsIs{David Rowe}
\item[Maintainer]\AsIs{David Rowe }\email{david@rowe-stats.com}\AsIs{}
\item[Description]\AsIs{Calculate the marginal likelihood using graphical evidence.}
\item[License]\AsIs{GPL (>= 2)}
\item[Imports]\AsIs{Rcpp, parallel, RcppXsimd}
\item[LinkingTo]\AsIs{Rcpp, RcppArmadillo}
\item[StagedInstall]\AsIs{no}
\item[RoxygenNote]\AsIs{7.3.1}
\item[Archs]\AsIs{x64}
\end{description}
\Rdcontents{\R{} topics documented:}
\inputencoding{utf8}
\HeaderA{graphicalEvidence-package}{A short title line describing what the package does}{graphicalEvidence.Rdash.package}
\aliasA{graphicalEvidence}{graphicalEvidence-package}{graphicalEvidence}
\keyword{package}{graphicalEvidence-package}
%
\begin{Description}\relax
A more detailed description of what the package does. A length
of about one to five lines is recommended.
\end{Description}
%
\begin{Details}\relax
This section should provide a more detailed overview of how to use the
package, including the most important functions.
\end{Details}
%
\begin{Author}\relax
Your Name, email optional.

Maintainer: Your Name <your@email.com>
\end{Author}
%
\begin{References}\relax
This optional section can contain literature or other references for
background information.
\end{References}
%
\begin{SeeAlso}\relax
Optional links to other man pages
\end{SeeAlso}
%
\begin{Examples}
\begin{ExampleCode}
  ## Not run: 
     ## Optional simple examples of the most important functions
     ## These can be in \dontrun{} and \donttest{} blocks.   
  
## End(Not run)
\end{ExampleCode}
\end{Examples}
\inputencoding{utf8}
\HeaderA{evidence}{Compute Marginal Likelihood using Graphical Evidence}{evidence}
%
\begin{Description}\relax
Computes the marginal likelihood of input data xx under one of the following
priors: Wishart, Bayesian Graphical Lasso (BGL), 
Graphical Horseshoe (GHS), and G-Wishart, specified under prior\_name. 
The number of runs is specified by num\_runs, where each run is by default
using a random permutation of the columns of xx, as marginal likelihood 
should be indepdendent of column permutation.
\end{Description}
%
\begin{Usage}
\begin{verbatim}
evidence(
  xx,
  burnin,
  nmc,
  prior_name = c("Wishart", "BGL", "GHS", "G_Wishart"),
  runs = 1,
  print_progress = FALSE,
  alpha = NULL,
  lambda = NULL,
  V = NULL,
  G = NULL
)
\end{verbatim}
\end{Usage}
%
\begin{Arguments}
\begin{ldescription}
\item[\code{xx}] The input data specified by a user for which the marginal 
likelihood is to be calculated. This should be input as a matrix like object
with each individual sample of xx representing one row.

\item[\code{burnin}] The number of iterations the MCMC sampler should iterate 
through and discard before beginning to save results.

\item[\code{nmc}] The number of samples that the MCMC sampler should use to estimate
quantities like posterior mean.

\item[\code{prior\_name}] The name of the prior for which the marginal should be 
calculated, this is one of 'Wishart', 'BGL', 'GHS', 'G\_Wishart'

\item[\code{runs}] The number of complete runs of the graphical evidence method that
will be executed. Specifying multiple runs allows estimation of the variance
of the estimator and by default will permute the columns of xx such that 
each run uses a random column ordering, as marginal likelihood should be 
independent of column permutations.

\item[\code{print\_progress}] A boolean which indicates whether progress should be 
displayed on the console as each row of the telescoping sum is compuated and
each run is completed.

\item[\code{alpha}] A number specifying alpha for the priors of 'Wishart' and 
'G\_Wishart'

\item[\code{lambda}] A number specifying lambda for the priors of 'BGL' and 'GHS'
prior

\item[\code{V}] The scale matrix when specifying 'Wishart' or 'G\_Wishart' prior

\item[\code{G}] The adjacency matrix when specifying 'G\_Wishart' prior
\end{ldescription}
\end{Arguments}
%
\begin{Value}
A list of results which contains the mean marginal likelihood, the
standard deviation of the estimator, and the raw results in a vector
\end{Value}
%
\begin{Examples}
\begin{ExampleCode}
# Compute the marginal 10 times with random column permutations of xx at each
# individual run for G-Wishart prior using 2,000 burnin and 10,000 sampled
# values at each call to the MCMC sampler
marginal_results <- evidence(
  xx, 2e3, 1e4, 'G_Wishart', 10, alpha=input_alpha, V=input_V, G=input_G
)
\end{ExampleCode}
\end{Examples}
\printindex{}
\end{document}
